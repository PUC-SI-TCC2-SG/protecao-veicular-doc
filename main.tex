\documentclass{article}
\usepackage[utf8]{inputenc}
\begin{document}


\section{Introdução}
XXXX é um aplicativo móvel de útilidade pública para todos aqueles que possuem automovéis e não possuem um tipo de proteção veícular. Nele é possível solicitar um serviço de emergência como guincho ou borracheiro, através de poucos cliques. Além disso, o aplicativo permite classificar os serviços prestados e os usuários solicitantes, para melhorar a experiência com futuros serviços, criando uma comunidade que busque cada vez mais aprimorar seus serviços e atendimentos. 

\section{Requisitos não funcionais}

\begin{table}[h!]
  \begin{center}
    \caption{Requisitos não funcionais.}
    \label{tab:table1}
    \begin{tabular}{|c|l|} 
      \hline
      \textbf{ID} & \textbf{Descrição}\\
      \hline
      RNF 001 & O aplicativo deve ser multiplataforma, funcionando             nos sistemas operacionais \\ 
              & Android e IOS. \\
      \hline
      RNF 002 & O aplicativo deve permitir autenticação pelas contas           do Facebook e do Google. \\
      \hline
      RNF 003 & O usuário deve ser capaz de solicitar um serviço de            emergência com no máximo \\
              & 3 interações com a tela. \\
      \hline
      RNF 004 & O tempo máximo de espera para que um prestador de              serviço aceite a solicitação \\
              & não deve exceder 1                minuto em horário    de pico, de 08:00 até 19:00. E não           exceder \\
              & 3 minutos no restante do dia. \\
     \hline
    \end{tabular}
  \end{center}
\end{table}

\section{Requisitos funcionais}

\begin{table}[h!]
  \begin{center}
    \caption{Requisitos funcionais.}
    \label{tab:table2}
    \begin{tabular}{|c|l|} 
      \hline
      \textbf{ID} & \textbf{Descrição}\\
      \hline
      RF 001 & Cadastro de usuário \\
      \hline
      RF 002 & Autenticação do usuário \\
      \hline
      RF 003 & Atualização de dados cadastrais do usuário \\
      \hline
      RF 004 & Adicionar método de pagamento \\
      \hline
      RF 005 & Buscar serviço de emergência \\
      \hline
      RF 006 & Aceitar serviço de emergência \\
      \hline
      RF 007 & Finalizar serviço \\
      \hline
      RF 008 & Avaliar profissional \\
      \hline
      RF 009 & Avaliar cliente \\
      \hline
      RF 010 & Visualizar histórico de serviços \\
      \hline
    \end{tabular}
  \end{center}
\end{table}

\begin{table}[h!]
  \begin{center}
    \label{tab:table3}
    \begin{tabular}{|l|l|}
      \hline
      \textbf{RF 001} & \textbf{Cadastro de usuário}\\
      \hline
      Objetivo          & Permitir que os usuários se cadastram no sistema. \\
      \hline
      Pré-requisitos    & Nenhum \\
      \hline
      Atores            & Cliente e Prestador de serviço \\
      \hline
      Fluxo principal   & 1 - Usuário clica em "Cadastrar" \\
                        & 2 - Sistema exibe formulário de cadastro \\
                        & 3 - Usuário preenche o formulário \\
                        & 4 - Usuário clica em "Registrar" \\
                        & 5 - Sistema exibe mensagem de sucesso \\
                        & 6 - Sistema envia e-mail para confirmação de registro \\
                        & 7 - Fim do caso de uso \\
      \hline
      Fluxo alternativo 1 & 1 - Usuário seleciona opção como                                  prestador de serviços \\
                          & 2 - Sistema exibe informações a mais          para preencher \\
      \hline
      Fluxo de exceção & 1 - Usuário não preenche todos os dados                           corretamente \\
                       & 2 - Sistema exibe mensagem de erro. \\
                       & 3 - Retornar ao Fluxo Principal Etapa 3. \\
      \hline
    \end{tabular}
  \end{center}
\end{table}

\begin{table}[h!]
  \begin{center}
    \label{tab:table4}
    \begin{tabular}{|l|l|}
      \hline
      \textbf{RF 002} & \textbf{Autenticação do usuário}\\
      \hline
      Objetivo          & Permitir que os usuários se autentique no sistema. \\
      \hline
      Pré-requisitos    & Ter efetuado registro no sistema \\
      \hline
      Atores            & Cliente e Prestador de serviço \\
      \hline
      Fluxo principal   & 1 - Sistema exibe formulário de login \\
                        & 2 - Usuário preenche formulário de login \\
                        & 2 - Sistema valida credenciais \\
                        & 3 - Sistema navega para a tela inicial do       sistema \\
                        & 4 - Fim do caso de uso \\
      \hline
      Fluxo alternativo 1 & 1 - Usuário seleciona "Entrar com Facebook" \\
                          & 2 - Sistema chama serviço de autenticação do Facebook \\
                          & 3 - Retorna para o Fluxo Principal Etapa 3 \\
      \hline
      Fluxo alternativo 2 & 1 - Usuário seleciona "Entrar com Google" \\
                          & 2 - Sistema chama serviço de autenticação do Google \\
                          & 3 - Retorna para o Fluxo Principal Etapa 3 \\
      \hline
      Fluxo de exceção & 1 - Usuário preenche dados incorretos \\
                       & 2 - Sistema exibe mensagem de erro. \\
                       & 3 - Retornar ao Fluxo Principal Etapa 1. \\
      \hline
    \end{tabular}
  \end{center}
\end{table}

\begin{table}[h!]
  \begin{center}
    \label{tab:table5}
    \begin{tabular}{|l|l|}
      \hline
      \textbf{RF 003} & \textbf{Atualização dos dados cadastrais do usuário}\\
      \hline
      Objetivo          & Permitir que os usuários altere os dados do seu perfil. \\
      \hline
      Pré-requisitos    & Usuário ter se autenticado no sistema \\
      \hline
      Atores            & Cliente e Prestador de serviço \\
      \hline
      Fluxo principal   & 1 - Usuário clica no menu e seleciona opção "Meus dados" \\
                        & 2 - Sistema exibe formulário com dados do usuário \\
                        & 3 - Usuário clica em "Editar" \\
                        & 4 - Sistema exibe formulário editável \\
                        & 5 - Usuário modifica os dados e clica em "Gravar"
                        & 6 - Fim do caso de uso \\
      \hline
      Fluxo de exceção & 1 - Usuário preenche dados incorretos \\
                       & 2 - Sistema exibe mensagem de erro. \\
                       & 3 - Retornar ao Fluxo Principal Etapa 5. \\
      \hline
    \end{tabular}
  \end{center}
\end{table}

\begin{table}[h!]
  \begin{center}
    \label{tab:table6}
    \begin{tabular}{|l|l|}
      \hline
      \textbf{RF 004} & \textbf{Adicionar método de pagamento}\\
      \hline
      Objetivo          & Permitir que o cliente adicione meios de pagamentos, cartões de crédito e fintechs. \\
      \hline
      Pré-requisitos    & Usuário ter se autenticado no sistema \\
      \hline
      Atores            & Cliente \\
      \hline
      Fluxo principal   & 1 - Usuário clica no menu e seleciona opção "Meios de pagamento" \\
                        & 2 - Sistema exibe formulário com dados a serem preenchidos \\
                        & 3 - Usuário insere informações \\
                        & 4 - Usuário clica em "Gravar"
                        & 5 - Fim do caso de uso \\
      \hline
      Fluxo de exceção & 1 - Usuário preenche dados incorretos \\
                       & 2 - Sistema exibe mensagem de erro. \\
                       & 3 - Retornar ao Fluxo Principal Etapa 3. \\
      \hline
    \end{tabular}
  \end{center}
\end{table}

\begin{table}[h!]
  \begin{center}
    \label{tab:table7}
    \begin{tabular}{|l|l|}
      \hline
      \textbf{RF 005} & \textbf{Buscar serviço de emergência}\\
      \hline
      Objetivo          & Permitir que o cliente busque um profissional que o auxilie em uma emergência com seu veículo. \\
      \hline
      Pré-requisitos    & Usuário ter se autenticado no sistema \\
      \hline
      Atores            & Cliente \\
      \hline
      Fluxo principal   & 1 - Usuário na tela inicial clica no serviço de sua preferência \\
                        & 2 - Sistema exibe mapa com sua localização atual \\
                        & 3 - Usuário confirma localização e método de pagamento \\
                        & 4 - Sistema envia notificação para prestadores de serviço.
                        & 5 - Fim do caso de uso \\
      \hline
      Fluxo alternativo 1 & 1 - GPS do dispositivo do Cliente desativado \\
                          & 2 - Sistema solicita acionamento do GPS do dispositivo \\
                          & 3 - Usuário aciona GPS \\
                          & 4 - Retornar ao Fluxo Principal Etapa 3 \\
      \hline
      Fluxo de exceção & 1 - Sistema não recebe nenhum aceite do pedido \\
                       & 2 - Sistema exibe mensagem que iniciará nova busca. \\
                       & 3 - Retornar ao Fluxo Principal Etapa 4. \\
      \hline
    \end{tabular}
  \end{center}
\end{table}

\begin{table}[h!]
  \begin{center}
    \label{tab:table8}
    \begin{tabular}{|l|l|}
      \hline
      \textbf{RF 006} & \textbf{Aceitar serviço de emergência}\\
      \hline
      Objetivo          & Permitir que o prestador de serviço aceite um chamado de emergência. \\
      \hline
      Pré-requisitos    & Usuário ter se autenticado no sistema e Cliente solicitar serviço \\
      \hline
      Atores            & Prestador de serviço \\
      \hline
      Fluxo principal   & 1 - Prestador recebe notificação de solicitação de serviço \\
                        & 2 - Sistema exibe no mapa localização do Cliente \\
                        & 3 - Prestador aceita solicitação de serviço \\
                        & 4 - Sistema envia notificação para o Cliente \\
                        & 5 - Sistema cria rota de navegação até o cliente.
                        & 6 - Fim do caso de uso \\
      \hline
      Fluxo alternativo 1   & 1 - Usuário rejeita serviço \\
                            & 2 - Sistema envia notificação para Cliente avisando sobre o cancelamento \\
                            & 3 - Sistema retorma as buscas por novos prestadores \\
                            & 4 - Fim do caso de uso \\
      \hline
    \end{tabular}
  \end{center}
\end{table}

\begin{table}[h!]
  \begin{center}
    \label{tab:table9}
    \begin{tabular}{|l|l|}
      \hline
      \textbf{RF 007} & \textbf{Finalizar serviço}\\
      \hline
      Objetivo          & Permitir que o prestador de serviço encerre o chamado de emergência. \\
      \hline
      Pré-requisitos    & Usuário ter se autenticado no sistema e Aceitar um serviço \\
      \hline
      Atores            & Prestador de serviço e Cliente \\
      \hline
      Fluxo principal   & 1 - Prestador clica em "Encerrar Serviço" \\
                        & 2 - Sistema exibe campo para preencher valor do serviço \\
                        & 3 - Prestador preenche valor e clicar em "Continuar" \\
                        & 4 - Sistema envia notificação para o Cliente confirmar valor do serviço.  \\
                        & 5 - Cliente confirma valor do serviço. \\
                        & 6 - Sistema encerra serviço. \\
                        & 7 - Fim do caso de uso \\
      \hline
      Fluxo de exceção 1    & 1 - Usuário rejeita valor do serviço \\
                            & 2 - Sistema envia notificação para o prestador avisando sobre o cancelamento \\
                            & 3 - Prestador informa novo valor \\
                            & 4 - Retorna para o Fluxo Principal Etapa 4 \\
      \hline
    \end{tabular}
  \end{center}
\end{table}

\begin{table}[h!]
  \begin{center}
    \label{tab:table10}
    \begin{tabular}{|l|l|}
      \hline
      \textbf{RF 008} & \textbf{Avaliar Profissional}\\
      \hline
      Objetivo          & Permitir que o cliente avalie os serviços prestados em três categorias. São elas: Atendimento, Preço e Velocidade \\
      \hline
      Pré-requisitos    & Usuário ter se autenticado no sistema e Prestador finalizar um serviço \\
      \hline
      Atores            & Cliente \\
      \hline
      Fluxo principal   & 1 - Sistema exibe formulário para o cliente avaliar o prestador de serviço. \\
                        & 2 - Usuário preenche formulário e envia \\
                        & 7 - Fim do caso de uso \\
      \hline
      Fluxo de alternativo 1    & 1 - Usuário não preenche o formulário \\
                                & 2 - Sistema fecha formulário \\
                                & 3 - Fim d caso de uso \\
      \hline
    \end{tabular}
  \end{center}
\end{table}

\begin{table}[h!]
  \begin{center}
    \label{tab:table11}
    \begin{tabular}{|l|l|}
      \hline
      \textbf{RF 009} & \textbf{Avaliar Cliente}\\
      \hline
      Objetivo          & Permitir que o prestador de serviço avalie o cliente. \\
      \hline
      Pré-requisitos    & Usuário ter se autenticado no sistema e Prestador finalizar um serviço \\
      \hline
      Atores            & Cliente \\
      \hline
      Fluxo principal   & 1 - Sistema exibe formulário para o cliente avaliar o prestador de serviço. \\
                        & 2 - Usuário preenche formulário e envia \\
                        & 7 - Fim do caso de uso \\
      \hline
      Fluxo de alternativo 1    & 1 - Usuário não preenche o formulário \\
                                & 2 - Sistema fecha formulário \\
                                & 3 - Fim d caso de uso \\
      \hline
    \end{tabular}
  \end{center}
\end{table}

\end{document}
